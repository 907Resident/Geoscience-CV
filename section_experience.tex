% YAAC Another Awesome CV LaTeX Template
%
% This template has been downloaded from:
% https://github.com/darwiin/yaac-another-awesome-cv
%
% Author:
% Christophe Roger
%
% Template license:
% CC BY-SA 4.0 (https://creativecommons.org/licenses/by-sa/4.0/)
%Section: Work Experience at the top
\sectionTitle{Professional Experience}{\faSuitcase}
%\renewcommand{\labelitemi}{$\bullet$}
\begin{experiences}
  \experience
    {Present}       {Long-term Continuous Observations of Gas Emissions in Upland Forest}{Oak Ridge}{TN}
    {May2018}      {
                    One of three projects of my dissertation:
                      \begin{itemize}
                        \item Working with NOAA’s Atmospheric Turbulence Diffusion Division (ATDD) to quantify \ce{CH4} emissions at soil-, stem-, canopy-atmosphere interfaces                        
                        \item Deployed CRDS (Picarro G2201-i) coupled to an autochamber (eosAC) and a homemade sleeve for soil- and stem-gas measurements, respectively
                        \item Plan to collect 6+ months (tens of GBs) of measurements               
                        \item Applying time-series \& signal processing analysis (wavelet \& spectral analysis)     
                      \end{itemize}
                    }
                    {MATLAB, Python, Electrical Eng., Chamber Fabrication, Continuous Measurements}
  \emptySeparator
  \experience
    {Present}       {Survey of Gas Emissions in Valles \& Yellowstone Calderas}{Sandoval Co., NM \&}{Yellowstone, WY}
    {Jul2017}      {
                    Two of three projects of my dissertation:
                      \begin{itemize}
                        \item Developed mobile laboratory using cavity ring-down spectroscopic technology to measure \ce{CH4} \& \ce{CO2} and carbon isotopes
                        \item Investigated spatial and temporal trends of gas emissions at both calderas to compare and contrast the two supervolcanoes      
                        \item Designed project to gather first measurements of diffuse \ce{CH4} emissions in N. American calderas 
                        \item Constructed an algorithm to model the sampling of soil gas emissions with linear programming and Monte Carlo simulation
                        \item Devised algorithms to quantify hundreds of flux and isotopic measurements at dozens of locations between both calderas
                        \item Recorded emissions prior, during, after geyser eruptions (including tallest active geyser, Steamboat)
                        \item Corresponded with hundreds of patrons during data collection
                      \end{itemize}
                    }
                    {MATLAB, Python, Lab to Field Instrument Deployment, Survey Measurements, Science Communication}
  \emptySeparator
  \experience
  {Aug2016}       {Surface Fluxes at Hydraulic Fracturing Sites}{Morgan County}{TN}
  {Oct2015}       {
                   Master's Thesis, Completed in 2016:
                     \begin{itemize}
                        \item Established methodology for spectroscopic-chamber (Picarro G2201-i \& eosAC) measurements at several hydraulic fracturing sites 
                        \item Sought to detect thermogenic \ce{CH4} from leaking natural gas wells with surface flux measurements and carbon isotope measurements
                        \item Fabricated static chambers to sample heavier hydrocarbons; processed samples with gas chromatograph          
                        \item Final products include data reduction algorithms and a manuscript                      
                      \end{itemize}
                    }
                    {Environmental Assessment, Energy Science, MATLAB, Gas Chromatography, Lab to Field Instrument Deployment}
  \emptySeparator
  \experience
  {May2014}        {Arctic Shrub Carbon Assimilation}{Toolik Lake}{AK}
  {Jan2013}        {
                    Senior Thesis, Completed in 2014:
                    \begin{itemize}
                        \item Traveled above the Arctic Circle to the tundra as a team member of a multi-institution NASA funded project      
                        \item Examined carbon assimilation by deciduous shrubs in a warming tundra           
                        \item Drafted and constructed homemade chamber in the lab and deployed the chamber and infrared gas analyzer in the field
                      \end{itemize}
                    }
                    {Open Path Infrared Gas Analysis, MATLAB, Chamber Fabrication} 
\end{experiences}
